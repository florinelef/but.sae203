\documentclass{report}
\usepackage[utf8x]{inputenc}
\begin{document}
Malori ALVAREZ, Florine LEFEBVRE \& Nathanaël COLLUMEAU

\section{Rapport SAE 2.03}\label{rapport-sae-2.03}

\ul{Sommaire :}

\begin{quote}
→ \hyperref[1S]{1\textsuperscript{ère} semaine}\\
Installation de l'OS, sudo, suppléments invités, distribution debian \&
automatisation\\
→ \hyperref[2S]{2\textsuperscript{ème} semaine}\\
Apprentissage du markdown, début des rapports sur StackEdit, réalisation
de l'archive (html, pdf, source et readme)\\
→ \hyperref[3S]{3\textsuperscript{ème} semaine}\\
Setup du git et choix de l'interface graphique\\
→ \hyperref[4S]{4\textsuperscript{ème} semaine}\\
Setup de Gitea\\
\end{quote}

\subsection{\texorpdfstring{1\textsuperscript{ère}
semaine}{1ère semaine}}\label{1S}

\subsubsection[ \hl{Création de la 1\textsuperscript{ère}
VM}]{\texorpdfstring{\protect\includegraphics[width=0.36458in,height=\textheight]{https://cdn-icons-png.flaticon.com/512/73/73575.png}
\hl{Création de la 1\textsuperscript{ère}
VM}}{drawing Création de la 1ère VM}}\label{drawing-cruxe9ation-de-la-1uxe8re-vm}

Tout d'abord, nous avons commencé en créant une nouvelle VM dans
\textbf{Oracle VM VirtualBox} que nous avons appelée
``\emph{SAE203auto}'', sans aucun fichier iso, mais en sélectionnant le
type (\textbf{Linux}) et la version (\textbf{Debian 12 Bookworm} en
version 64-bit). Par la suite, nous avons continué la création de la VM
en laissant les paramètres par défaut. Cependant, il manquait le fichier
iso, nous avons donc été dans la configuration de la VM, dans l'onglet
\texttt{Storage}, puis dans ``\texttt{Controller:\ IDE}''
-\textgreater{} ``\texttt{Empty}'' pour rajouter dans
``\texttt{IDE\ Primary\ Device\ 0}'' le fichier
``\textbf{S203-Debian12.viso}'' afin que les fichiers de configuration
automatique se lancent convenablement.

\includegraphics{https://lh7-us.googleusercontent.com/HwGyxgCVJ4FucrncIYnrRtDJVPEdrYU__IEii__UMg29YSwengtuclEehw575ZxLd2NAXWbh1VpI0xNDq_o1kRc3V7HD2PmTYJKxIz14KGTN7T8BkUGBZ_M6d1aJU1k5DbnO3vo9U3_ELN5BtngKaZY}\\
Avant de lancer la VM, nous avons modifié les fichiers
``\texttt{vboxpostinstall.sh}'' et ``\texttt{preseed-fr.cfg}'' pour
paramétrer notre VM, ainsi que l'uuid dans le fichier viso.

Dans la \textbf{preseed}, on a modifié la ligne 83 afin d'avoir
l'installation de mate :

\texttt{tasksel\ tasksel/first\ multiselect\ standard\ ssh-server,\ mate-desktop}\strut \\

\textbf{\includegraphics{https://lh7-us.googleusercontent.com/NXvSVVd-LuTqc8ObAwmz6XIFKAmHAJZ_OK8OafUy8bNNJKmjDhpUu9oPERd58B_ClqZOfV2UAREMnBZijqGrVuKCphyTQjMiojFAJzsT8ltba8KWeEGLdtXVldk-NC7NrDMONRHsnqa2n-hrjT5Un5s}}\\
Dans le \textbf{vboxpostinstall}, nous avons ajouté dans la partie ``Run
user command'' les commandes pour installer les fonctionnalités
attendues :\\
\textbf{\includegraphics{https://lh7-us.googleusercontent.com/UNWV9n-fmQzs8ckQ6X1Jv6B-sqRiQcOuRqqGqFoyHddPlooD2WQCCO5Nn3KV80svOt0gnpqwU0Pc20U3ihF7d2KU3x-EEJvqRnc7PBaEbutOic7IBDlwmiBTQYdecS9YvDb325QQ2BqxfhV5BRyTX68}}\\
\texttt{log\_command\_in\_target\ usermod\ -a\ -G\ sudo\ user}

\texttt{log\_command\_in\_target\ apt-get\ install\ sudo}

\texttt{log\_command\_in\_target\ sudo\ apt-get\ install\ git}

\texttt{log\_command\_in\_target\ sudo\ apt-get\ install\ sqlite3}

\texttt{log\_command\_in\_target\ sudo\ apt-get\ install\ curl}

\texttt{log\_command\_in\_target\ sudo\ apt-get\ install\ bash-completion}

\texttt{log\_command\_in\_target\ sudo\ apt-get\ install\ neofetch}

Nous avons rencontré quelques soucis à l'installation à cause de ces
commandes, mais en commentant les 5 dernières lignes \emph{( sudo
apt-get\ldots)} l'ensemble refonctionnait de nouveau, mais sans les
installations.\\
Nous avons plus tard réalisé que le problème venait du fait que la
commande \texttt{sudo} attend une entrée utilisateur, ce qui bloquait
complètement l'exécution du script. De plus, ce script s'exécute déjà en
tant que root, alors il nous a suffit de retirer \texttt{sudo} de toutes
les commandes pour que le script fonctionne.

\subsubsection[ \hl{Questions} : Configuration matérielle dans
VirtualBox]{\texorpdfstring{\protect\includegraphics[width=0.20833in,height=\textheight]{https://upload.wikimedia.org/wikipedia/commons/thumb/f/f8/Question_mark_alternate.svg/788px-Question_mark_alternate.svg.png}
\hl{Questions} : Configuration matérielle dans
VirtualBox}{drawing Questions : Configuration matérielle dans VirtualBox}}\label{drawing-questions-configuration-matuxe9rielle-dans-virtualbox}

\begin{description}
\tightlist
\item[Que signifie ``64-bit'' dans ``Debian 64-bit'' ?]
\begin{itemize}
\tightlist
\item[]
\item
  C'est la largeur des registres soit la façon dont le processeur gère
  les informations qu'il doit effectuer.
\end{itemize}
\item[Quelle est la configuration réseau utilisée par défaut ?]
\begin{itemize}
\tightlist
\item[]
\item
  La configuration réseau utilisée par défaut est la configuration en
  mode \emph{NAT}.
\end{itemize}
\item[Quel est le nom du fichier XML contenant la configuration de votre
machine ?]
\begin{itemize}
\tightlist
\item[]
\item
  Le nom du fichier XML contenant la configuration de votre machine est
  le fichier \texttt{debian.xml}
\end{itemize}
\item[Comment vous modifieriez ce fichier de configuration pour mettre 2
processeurs à votre machine ?]
\begin{itemize}
\tightlist
\item[]
\item
  Pour mettre 2 processeurs à votre machine, je modifierais le fichier
  de configuration en ajoutant la ligne suivante :
  \texttt{\textless{}vcpu\ placement=\textquotesingle{}static\textquotesingle{}\textgreater{}2\textless{}/vcpu\textgreater{}.}
\end{itemize}
\end{description}

\subsubsection[ \hl{Questions} : Installation OS de
base]{\texorpdfstring{\protect\includegraphics[width=0.20833in,height=\textheight]{https://upload.wikimedia.org/wikipedia/commons/thumb/f/f8/Question_mark_alternate.svg/788px-Question_mark_alternate.svg.png}
\hl{Questions} : Installation OS de
base}{drawing Questions : Installation OS de base}}\label{drawing-questions-installation-os-de-base}

\begin{description}
\tightlist
\item[Qu'est-ce qu'un fichier iso bootable ?]
\begin{itemize}
\tightlist
\item[]
\item
  Un fichier iso bootable est un fichier qui s'exécute au démarrage de
  la VM et qui contient une instance d'un système d'exploitation. Il
  permet de créer la machine.
\end{itemize}
\item[Qu'est-ce que MATE ? GNOME ?]
\begin{itemize}
\tightlist
\item[]
\item
  \emph{Mate} et \emph{Gnome} sont des environnements graphiques de
  debian
\end{itemize}
\item[Qu'est-ce qu'un serveur web ?]
\begin{itemize}
\tightlist
\item[]
\item
  Un serveur web est un logiciel qui permet de stocker et de diffuser
  des pages web.
\end{itemize}
\item[Qu'est-ce qu'un serveur ssh ?]
\begin{itemize}
\tightlist
\item[]
\item
  SSH est un protocole de communication sécurisé qui permet d'accéder à
  distance à un serveur.
\end{itemize}
\item[Qu'est-ce qu'un serveur mandataire ?]
\begin{itemize}
\tightlist
\item[]
\item
  Un serveur mandataire est un serveur qui permet de filtrer les
  requêtes des clients.
\end{itemize}
\end{description}

\subsubsection[ \hl{Questions} :
sudo]{\texorpdfstring{\protect\includegraphics[width=0.20833in,height=\textheight]{https://upload.wikimedia.org/wikipedia/commons/thumb/f/f8/Question_mark_alternate.svg/788px-Question_mark_alternate.svg.png}
\hl{Questions} :
sudo}{drawing Questions : sudo}}\label{drawing-questions-sudo}

\begin{description}
\tightlist
\item[Comment peux-ton savoir à quels groupes appartient l'utilisateur
user ?]
\begin{itemize}
\tightlist
\item[]
\item
  On utilise la commande : \$ groups user
\end{itemize}
\end{description}

\subsubsection[ \hl{Questions} : Suppléments
invités]{\texorpdfstring{\protect\includegraphics[width=0.20833in,height=\textheight]{https://upload.wikimedia.org/wikipedia/commons/thumb/f/f8/Question_mark_alternate.svg/788px-Question_mark_alternate.svg.png}
\hl{Questions} : Suppléments
invités}{drawing Questions : Suppléments invités}}\label{drawing-questions-suppluxe9ments-invituxe9s}

\begin{description}
\tightlist
\item[Quelle est la version du noyau Linux utilisé par votre VM ?
N'oubliez pas, comme pour toutes les questions, de justifier votre
réponse.]
\begin{itemize}
\tightlist
\item[]
\item
  On l'obtient avec la commande uname -r, uname donne des informations
  sur le système qu'on utilise. Ici il s'agit de amd64.
\end{itemize}
\item[À quoi servent les suppléments invités ? Donner 2 principales
raisons de les installer.]
\begin{itemize}
\tightlist
\item[]
\item
  Les suppléments invités servent à :\\
  -\textgreater{} améliorer les performances de la machine virtuelle\\
  -\textgreater{} faciliter le partage de fichiers entre la machine
  virtuelle et l'hôte.
\end{itemize}
\item[À quoi sert la commande mount (dans notre cas de figure et dans le
cas général) ?]
\begin{itemize}
\tightlist
\item[]
\item
  Elle permet de monter une partition sur le disque dans le cas général.
\end{itemize}
\end{description}

\subsubsection[ \hl{Questions} : Quelques
questions]{\texorpdfstring{\protect\includegraphics[width=0.20833in,height=\textheight]{https://upload.wikimedia.org/wikipedia/commons/thumb/f/f8/Question_mark_alternate.svg/788px-Question_mark_alternate.svg.png}
\hl{Questions} : Quelques
questions}{drawing Questions : Quelques questions}}\label{drawing-questions-quelques-questions}

\begin{description}
\tightlist
\item[Il existe 3 durées de prise en charge (support) de ces versions :
la durée minimale la durée en support long terme (LTS) et la durée en
support long terme étendue (ELTS). Quelles sont les durées de ces prises
en charge ?]
\begin{itemize}
\tightlist
\item[]
\item
  La durée minimale, de support long terme (LTS) et celle en support
  long terme étendue (ELTS) sont de 5 ans
\end{itemize}
\item[Pendant combien de temps les mises à jour de sécurité seront-elles
fournies ?]
\begin{itemize}
\tightlist
\item[]
\item
  Les mises à jour de sécurité seront fournies pendant 5 ans.
\end{itemize}
\item[Combien de version au minimum sont activement maintenues par
Debian ? Donnez leur nom générique (= les types de distribution).]
\begin{itemize}
\tightlist
\item[]
\item
  Il y a 3 versions au minimum activement maintenues par Debian :
  \emph{stable, testing} et \emph{unstable}.
\end{itemize}
\item[Chaque distribution majeur possède un nom de code différent. Par
exemple la version majeur actuelle (Debian 12) se nomme bookworm. D'où
viennent les noms de code données aux distributions ?]
\begin{itemize}
\tightlist
\item[]
\item
  Les noms de code donnés aux distributions viennent de personnages de
  Toy Story. La décision d'utiliser des noms provenant de Toy Story a
  été prise par Bruce Perens qui était, à l'époque, responsable du
  projet Debian et travaillait chez Pixar, la société qui a produit les
  films.
\end{itemize}
\item[L'un des atouts de Debian fut le nombre d'architecture (≈
processeurs) officiellement prises en charge. Combien et lesquelles sont
prises en charge par la version Bullseye ?]
\begin{itemize}
\tightlist
\item[]
\item
  Il y a 10 architectures officiellement prises en charge par la version
  Bullseye : amd64 (la notre) arm64 armel armhf i386 mips mips64el
  mipsel ppc64el s390x.
\end{itemize}
\item[Quel était le numéro de version de cette distribution ?]
\begin{itemize}
\tightlist
\item[]
\item
  Le numéro de version de cette distribution était 11.
\end{itemize}
\item[Quel a été le premier nom de code utilisé ?]
\begin{itemize}
\tightlist
\item[]
\item
  Le premier nom de code utilisé était Buzz.
\end{itemize}
\end{description}

\subsection{\texorpdfstring{2\textsuperscript{ème}
semaine}{2ème semaine}}\label{2S}

Nous nous sommes initié au \textbf{Markdown} via les sites qui nous ont
été fournis pour s'exercer
(\href{http://markdowntutorial.com}{markdowntutorial.com} \&
\href{http://commonmark.org}{commonmark.org}) mais certain.\emph{es}
d'entre nous savaient déjà l'utiliser par le passé.

Pour le choix de l'éditeur, nous préférons sur navigateur afin de
s'épargner de devoir installer un logiciel sur chaque machines où nous
travaillons. Nous avons donc choisi \textbf{StackEdit} car après avoir
essayé d'autres options telles que \textbf{NextCloud} ou
\textbf{github/lab}, celui-ci nous paraissait optimal bien qu'on ne
puisse pas rédiger le rapport en temps réel ensemble comme sur
\textbf{NextCloud}.

Nous avons exporté notre rapport sous le format .md et nous l'avons
compilé en HTML avec la commande:

\texttt{pandoc.exe\ -\/-standalone\ -\/-template\ template-compilation-s203.html\ rapport-s203-markdown.md\ -o\ rapport-s203.html}

L'utilisation de l'option -\/-template permet, par exemple, de donner un
titre à la page et d'utiliser une feuille de style CSS externe.

En revanche, nous nous sommes heurté.e.s à des obstacles pour la
compilation au format PDF avec pdflatex.

\subsection{\texorpdfstring{3\textsuperscript{ème}
semaine}{3ème semaine}}\label{3S}

\subsubsection[ \hl{Questions} : Interfaces graphiques pour
\emph{git}]{\texorpdfstring{\protect\includegraphics[width=0.20833in,height=\textheight]{https://upload.wikimedia.org/wikipedia/commons/thumb/f/f8/Question_mark_alternate.svg/788px-Question_mark_alternate.svg.png}
\hl{Questions} : Interfaces graphiques pour
\emph{git}}{drawing Questions : Interfaces graphiques pour git}}\label{drawing-questions-interfaces-graphiques-pour-git}

\begin{description}
\tightlist
\item[Qu'est-ce que le logiciel gitk ? Comment se lance-t-il ?]
\begin{itemize}
\tightlist
\item[]
\item
  gitk est un navigateur de dépôt graphique qui permet de voir nos
  branches et nos commits, pulls, push, etc. sur git. Il se lance depuis
  le terminal depuis un dépôt.
\end{itemize}
\item[Qu'est-ce que le logiciel git-gui ? Comment se lance-t-il ?]
\begin{itemize}
\tightlist
\item[]
\item
  Interface graphique de Git basée sur Tcl/Tk. Git gui permet aux
  utilisateurs d'apporter des modifications à leur dépôt en faisant de
  nouveaux commits, en modifiant les commits existants, en créant des
  branches, en effectuant des fusions locales, et en récupérant/poussant
  vers des dépôts distants.
\end{itemize}

\begin{itemize}
\tightlist
\item
  Contrairement à gitk, git gui se concentre sur la génération de commit
  et l'annotation de fichiers uniques et n'affiche pas l'historique du
  projet. Il fournit cependant des actions de menu pour démarrer une
  session gitk à partir de git gui.
\end{itemize}
\item[Pourquoi avez-vous choisi ce logiciel ?]
\begin{itemize}
\tightlist
\item[]
\item
  Nous avons choisi \href{https://www.gitkraken.com/}{GitKraken} car son
  interface semble plus logique, intuitive et esthéthiquement plaisante.
\end{itemize}
\item[Comment l'avez vous installé ?]
\begin{itemize}
\tightlist
\item[]
\item
  Sur leur site avec la version demo Linux-Debian. L'essai gratuit
  couvre entièrement notre temps d'utilisation du logiciel pour la SAE.
\end{itemize}
\item[Comparer-le aux outils inclus avec git (et installé précédemment)
ainsi qu'avec ce qui serait fait 2/3 \textbar{} 1.1. Configuration
globale de git en ligne de commande pure : fonctionnalités avantages,
inconvénients\ldots{}]
\begin{itemize}
\tightlist
\item[]
\item
  L'interface est logique (schéma clairs, différentes couleurs,
  interface simpliste.) et nous permet de visualiser simplement
  l'arborescence, nous faisant gagner énormément de temps. Des outils
  nous permettent de visualiser les modifications rapidement en passant
  le curseur dessus, on dispose également de stats par utilisateurs et
  d'actions effectuables simultanément sur plusieurs repository
  sélectionnés (fetch, pull, clone ect).
\end{itemize}
\end{description}

\subsection{\texorpdfstring{4\textsuperscript{ème}
semaine}{4ème semaine}}\label{4S}

\subsubsection[ \hl{Questions} : Qu'est ce que \emph{Gitea}
?]{\texorpdfstring{\protect\includegraphics[width=0.20833in,height=\textheight]{https://upload.wikimedia.org/wikipedia/commons/thumb/f/f8/Question_mark_alternate.svg/788px-Question_mark_alternate.svg.png}
\hl{Questions} : Qu'est ce que \emph{Gitea}
?}{drawing Questions : Qu'est ce que Gitea ?}}\label{drawing-questions-quest-ce-que-gitea}

Gitea est une forge logicielle libre en Go sous licence MIT, pour
l'hébergement de développement logiciel, basé sur le logiciel de gestion
de versions Git pour la gestion du code source, comportant un système de
suivi des bugs, un wiki, ainsi que des outils pour la relecture de code.
(Source Wikipédia)

\subsubsection[ \hl{Questions} : À quels logiciels peut-on le comparer
?]{\texorpdfstring{\protect\includegraphics[width=0.20833in,height=\textheight]{https://upload.wikimedia.org/wikipedia/commons/thumb/f/f8/Question_mark_alternate.svg/788px-Question_mark_alternate.svg.png}
\hl{Questions} : À quels logiciels peut-on le comparer
?}{drawing Questions : À quels logiciels peut-on le comparer ?}}\label{drawing-questions-uxe0-quels-logiciels-peut-on-le-comparer}

\subsubsection[ \hl{Installation de
Gitea}]{\texorpdfstring{\protect\includegraphics[width=0.36458in,height=\textheight]{https://cdn-icons-png.flaticon.com/512/73/73575.png}
\hl{Installation de
Gitea}}{drawing Installation de Gitea}}\label{drawing-installation-de-gitea}

\begin{itemize}
\tightlist
\item
  On commence par ajouter la règle de redirection de port
\end{itemize}

\includegraphics{https://cdn.discordapp.com/attachments/1072839621567856690/1218651902045327391/Sans_titre.png?ex=66087106&is=65f5fc06&hm=9b8d57958b4eb5c5e24f237b782bed7447981b2bdedd1f91e9192ae1eb2f80bb&}

\begin{itemize}
\tightlist
\item
  On utilise wget avec le lien de téléchargement associé à notre version
  de linux, en l'occurrence : linux-amd64 (cf.~Semaine 06 - Question 4)
\end{itemize}

\texttt{\textgreater{}\ wget\ -O\ gitea\ https://dl.gitea.com/gitea/1.21.7/gitea-1.21.7-linux-amd64}

\texttt{\textgreater{}\ chmod\ +x\ gitea}

\begin{itemize}
\item
  Télécharger le fichier asc :
  \url{https://dl.gitea.com/gitea/1.21.7/gitea-1.21.7-linux-amd64.asc}
\item
  Ajout des clés binaires
\end{itemize}

\texttt{\textgreater{}\ gpg\ -\/-keyserver\ keys.openpgp.org\ -\/-recv\ 7C9E68152594688862D62AF62D9AE806EC1592E2}

\texttt{\textgreater{}\ gpg\ -\/-verify\ gitea-1.21.7-linux-amd64.asc\ gitea}

\begin{itemize}
\tightlist
\item
  Création d'un utilisateur ``git''
\end{itemize}

\texttt{\textgreater{}adduser\ \textbackslash{}\ -\/-system\ \textbackslash{}\ -\/-shell\ /bin/bash\ \textbackslash{}\ -\/-gecos\ \textquotesingle{}Git\ Version\ Control\textquotesingle{}\ \textbackslash{}\ -\/-group\ \textbackslash{}\ -\/-disabled-password\ \textbackslash{}\ -\/-home\ /home/git\ \textbackslash{}\ git}

\begin{itemize}
\tightlist
\item
  Création des dossiers nécessaires
\end{itemize}

\texttt{\textgreater{}\ mkdir\ -p\ /var/lib/gitea/\{custom,data,log\}}

\texttt{chown\ -R\ git:git\ /var/lib/gitea/}

\texttt{chmod\ -R\ 750\ /var/lib/gitea/}

\texttt{mkdir\ /etc/gitea}

\texttt{chown\ root:git\ /etc/gitea}

\texttt{chmod\ 770\ /etc/gitea}

\end{document}